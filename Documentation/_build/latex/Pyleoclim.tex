% Generated by Sphinx.
\def\sphinxdocclass{report}
\documentclass[letterpaper,10pt,english]{sphinxmanual}
\usepackage[utf8]{inputenc}
\DeclareUnicodeCharacter{00A0}{\nobreakspace}
\usepackage{cmap}
\usepackage[T1]{fontenc}
\usepackage{babel}
\usepackage{times}
\usepackage[Bjarne]{fncychap}
\usepackage{longtable}
\usepackage{sphinx}
\usepackage{multirow}
\usepackage{eqparbox}
\usepackage{amsfonts}

\addto\captionsenglish{\renewcommand{\figurename}{Fig. }}
\addto\captionsenglish{\renewcommand{\tablename}{Table }}
\SetupFloatingEnvironment{literal-block}{name=Listing }



\title{Pyleoclim Documentation}
\date{February 14, 2017}
\release{0.1.4}
\author{Deborah Khider, Julien Emile-Geay}
\newcommand{\sphinxlogo}{}
\renewcommand{\releasename}{Release}
\setcounter{tocdepth}{2}
\makeindex

\makeatletter
\def\PYG@reset{\let\PYG@it=\relax \let\PYG@bf=\relax%
    \let\PYG@ul=\relax \let\PYG@tc=\relax%
    \let\PYG@bc=\relax \let\PYG@ff=\relax}
\def\PYG@tok#1{\csname PYG@tok@#1\endcsname}
\def\PYG@toks#1+{\ifx\relax#1\empty\else%
    \PYG@tok{#1}\expandafter\PYG@toks\fi}
\def\PYG@do#1{\PYG@bc{\PYG@tc{\PYG@ul{%
    \PYG@it{\PYG@bf{\PYG@ff{#1}}}}}}}
\def\PYG#1#2{\PYG@reset\PYG@toks#1+\relax+\PYG@do{#2}}

\expandafter\def\csname PYG@tok@nf\endcsname{\def\PYG@tc##1{\textcolor[rgb]{0.02,0.16,0.49}{##1}}}
\expandafter\def\csname PYG@tok@sc\endcsname{\def\PYG@tc##1{\textcolor[rgb]{0.25,0.44,0.63}{##1}}}
\expandafter\def\csname PYG@tok@go\endcsname{\def\PYG@tc##1{\textcolor[rgb]{0.20,0.20,0.20}{##1}}}
\expandafter\def\csname PYG@tok@nt\endcsname{\let\PYG@bf=\textbf\def\PYG@tc##1{\textcolor[rgb]{0.02,0.16,0.45}{##1}}}
\expandafter\def\csname PYG@tok@err\endcsname{\def\PYG@bc##1{\setlength{\fboxsep}{0pt}\fcolorbox[rgb]{1.00,0.00,0.00}{1,1,1}{\strut ##1}}}
\expandafter\def\csname PYG@tok@cp\endcsname{\def\PYG@tc##1{\textcolor[rgb]{0.00,0.44,0.13}{##1}}}
\expandafter\def\csname PYG@tok@nc\endcsname{\let\PYG@bf=\textbf\def\PYG@tc##1{\textcolor[rgb]{0.05,0.52,0.71}{##1}}}
\expandafter\def\csname PYG@tok@mo\endcsname{\def\PYG@tc##1{\textcolor[rgb]{0.13,0.50,0.31}{##1}}}
\expandafter\def\csname PYG@tok@gh\endcsname{\let\PYG@bf=\textbf\def\PYG@tc##1{\textcolor[rgb]{0.00,0.00,0.50}{##1}}}
\expandafter\def\csname PYG@tok@kt\endcsname{\def\PYG@tc##1{\textcolor[rgb]{0.56,0.13,0.00}{##1}}}
\expandafter\def\csname PYG@tok@m\endcsname{\def\PYG@tc##1{\textcolor[rgb]{0.13,0.50,0.31}{##1}}}
\expandafter\def\csname PYG@tok@nb\endcsname{\def\PYG@tc##1{\textcolor[rgb]{0.00,0.44,0.13}{##1}}}
\expandafter\def\csname PYG@tok@sh\endcsname{\def\PYG@tc##1{\textcolor[rgb]{0.25,0.44,0.63}{##1}}}
\expandafter\def\csname PYG@tok@gd\endcsname{\def\PYG@tc##1{\textcolor[rgb]{0.63,0.00,0.00}{##1}}}
\expandafter\def\csname PYG@tok@gu\endcsname{\let\PYG@bf=\textbf\def\PYG@tc##1{\textcolor[rgb]{0.50,0.00,0.50}{##1}}}
\expandafter\def\csname PYG@tok@sb\endcsname{\def\PYG@tc##1{\textcolor[rgb]{0.25,0.44,0.63}{##1}}}
\expandafter\def\csname PYG@tok@mb\endcsname{\def\PYG@tc##1{\textcolor[rgb]{0.13,0.50,0.31}{##1}}}
\expandafter\def\csname PYG@tok@ss\endcsname{\def\PYG@tc##1{\textcolor[rgb]{0.32,0.47,0.09}{##1}}}
\expandafter\def\csname PYG@tok@kd\endcsname{\let\PYG@bf=\textbf\def\PYG@tc##1{\textcolor[rgb]{0.00,0.44,0.13}{##1}}}
\expandafter\def\csname PYG@tok@sx\endcsname{\def\PYG@tc##1{\textcolor[rgb]{0.78,0.36,0.04}{##1}}}
\expandafter\def\csname PYG@tok@kn\endcsname{\let\PYG@bf=\textbf\def\PYG@tc##1{\textcolor[rgb]{0.00,0.44,0.13}{##1}}}
\expandafter\def\csname PYG@tok@c\endcsname{\let\PYG@it=\textit\def\PYG@tc##1{\textcolor[rgb]{0.25,0.50,0.56}{##1}}}
\expandafter\def\csname PYG@tok@kr\endcsname{\let\PYG@bf=\textbf\def\PYG@tc##1{\textcolor[rgb]{0.00,0.44,0.13}{##1}}}
\expandafter\def\csname PYG@tok@bp\endcsname{\def\PYG@tc##1{\textcolor[rgb]{0.00,0.44,0.13}{##1}}}
\expandafter\def\csname PYG@tok@gi\endcsname{\def\PYG@tc##1{\textcolor[rgb]{0.00,0.63,0.00}{##1}}}
\expandafter\def\csname PYG@tok@s1\endcsname{\def\PYG@tc##1{\textcolor[rgb]{0.25,0.44,0.63}{##1}}}
\expandafter\def\csname PYG@tok@ge\endcsname{\let\PYG@it=\textit}
\expandafter\def\csname PYG@tok@gp\endcsname{\let\PYG@bf=\textbf\def\PYG@tc##1{\textcolor[rgb]{0.78,0.36,0.04}{##1}}}
\expandafter\def\csname PYG@tok@se\endcsname{\let\PYG@bf=\textbf\def\PYG@tc##1{\textcolor[rgb]{0.25,0.44,0.63}{##1}}}
\expandafter\def\csname PYG@tok@mi\endcsname{\def\PYG@tc##1{\textcolor[rgb]{0.13,0.50,0.31}{##1}}}
\expandafter\def\csname PYG@tok@cm\endcsname{\let\PYG@it=\textit\def\PYG@tc##1{\textcolor[rgb]{0.25,0.50,0.56}{##1}}}
\expandafter\def\csname PYG@tok@ni\endcsname{\let\PYG@bf=\textbf\def\PYG@tc##1{\textcolor[rgb]{0.84,0.33,0.22}{##1}}}
\expandafter\def\csname PYG@tok@nl\endcsname{\let\PYG@bf=\textbf\def\PYG@tc##1{\textcolor[rgb]{0.00,0.13,0.44}{##1}}}
\expandafter\def\csname PYG@tok@kc\endcsname{\let\PYG@bf=\textbf\def\PYG@tc##1{\textcolor[rgb]{0.00,0.44,0.13}{##1}}}
\expandafter\def\csname PYG@tok@nv\endcsname{\def\PYG@tc##1{\textcolor[rgb]{0.73,0.38,0.84}{##1}}}
\expandafter\def\csname PYG@tok@gt\endcsname{\def\PYG@tc##1{\textcolor[rgb]{0.00,0.27,0.87}{##1}}}
\expandafter\def\csname PYG@tok@cpf\endcsname{\let\PYG@it=\textit\def\PYG@tc##1{\textcolor[rgb]{0.25,0.50,0.56}{##1}}}
\expandafter\def\csname PYG@tok@kp\endcsname{\def\PYG@tc##1{\textcolor[rgb]{0.00,0.44,0.13}{##1}}}
\expandafter\def\csname PYG@tok@c1\endcsname{\let\PYG@it=\textit\def\PYG@tc##1{\textcolor[rgb]{0.25,0.50,0.56}{##1}}}
\expandafter\def\csname PYG@tok@nn\endcsname{\let\PYG@bf=\textbf\def\PYG@tc##1{\textcolor[rgb]{0.05,0.52,0.71}{##1}}}
\expandafter\def\csname PYG@tok@s\endcsname{\def\PYG@tc##1{\textcolor[rgb]{0.25,0.44,0.63}{##1}}}
\expandafter\def\csname PYG@tok@il\endcsname{\def\PYG@tc##1{\textcolor[rgb]{0.13,0.50,0.31}{##1}}}
\expandafter\def\csname PYG@tok@vc\endcsname{\def\PYG@tc##1{\textcolor[rgb]{0.73,0.38,0.84}{##1}}}
\expandafter\def\csname PYG@tok@k\endcsname{\let\PYG@bf=\textbf\def\PYG@tc##1{\textcolor[rgb]{0.00,0.44,0.13}{##1}}}
\expandafter\def\csname PYG@tok@s2\endcsname{\def\PYG@tc##1{\textcolor[rgb]{0.25,0.44,0.63}{##1}}}
\expandafter\def\csname PYG@tok@w\endcsname{\def\PYG@tc##1{\textcolor[rgb]{0.73,0.73,0.73}{##1}}}
\expandafter\def\csname PYG@tok@nd\endcsname{\let\PYG@bf=\textbf\def\PYG@tc##1{\textcolor[rgb]{0.33,0.33,0.33}{##1}}}
\expandafter\def\csname PYG@tok@cs\endcsname{\def\PYG@tc##1{\textcolor[rgb]{0.25,0.50,0.56}{##1}}\def\PYG@bc##1{\setlength{\fboxsep}{0pt}\colorbox[rgb]{1.00,0.94,0.94}{\strut ##1}}}
\expandafter\def\csname PYG@tok@si\endcsname{\let\PYG@it=\textit\def\PYG@tc##1{\textcolor[rgb]{0.44,0.63,0.82}{##1}}}
\expandafter\def\csname PYG@tok@gs\endcsname{\let\PYG@bf=\textbf}
\expandafter\def\csname PYG@tok@o\endcsname{\def\PYG@tc##1{\textcolor[rgb]{0.40,0.40,0.40}{##1}}}
\expandafter\def\csname PYG@tok@ow\endcsname{\let\PYG@bf=\textbf\def\PYG@tc##1{\textcolor[rgb]{0.00,0.44,0.13}{##1}}}
\expandafter\def\csname PYG@tok@vi\endcsname{\def\PYG@tc##1{\textcolor[rgb]{0.73,0.38,0.84}{##1}}}
\expandafter\def\csname PYG@tok@gr\endcsname{\def\PYG@tc##1{\textcolor[rgb]{1.00,0.00,0.00}{##1}}}
\expandafter\def\csname PYG@tok@mf\endcsname{\def\PYG@tc##1{\textcolor[rgb]{0.13,0.50,0.31}{##1}}}
\expandafter\def\csname PYG@tok@sd\endcsname{\let\PYG@it=\textit\def\PYG@tc##1{\textcolor[rgb]{0.25,0.44,0.63}{##1}}}
\expandafter\def\csname PYG@tok@vg\endcsname{\def\PYG@tc##1{\textcolor[rgb]{0.73,0.38,0.84}{##1}}}
\expandafter\def\csname PYG@tok@ch\endcsname{\let\PYG@it=\textit\def\PYG@tc##1{\textcolor[rgb]{0.25,0.50,0.56}{##1}}}
\expandafter\def\csname PYG@tok@mh\endcsname{\def\PYG@tc##1{\textcolor[rgb]{0.13,0.50,0.31}{##1}}}
\expandafter\def\csname PYG@tok@no\endcsname{\def\PYG@tc##1{\textcolor[rgb]{0.38,0.68,0.84}{##1}}}
\expandafter\def\csname PYG@tok@na\endcsname{\def\PYG@tc##1{\textcolor[rgb]{0.25,0.44,0.63}{##1}}}
\expandafter\def\csname PYG@tok@ne\endcsname{\def\PYG@tc##1{\textcolor[rgb]{0.00,0.44,0.13}{##1}}}
\expandafter\def\csname PYG@tok@sr\endcsname{\def\PYG@tc##1{\textcolor[rgb]{0.14,0.33,0.53}{##1}}}

\def\PYGZbs{\char`\\}
\def\PYGZus{\char`\_}
\def\PYGZob{\char`\{}
\def\PYGZcb{\char`\}}
\def\PYGZca{\char`\^}
\def\PYGZam{\char`\&}
\def\PYGZlt{\char`\<}
\def\PYGZgt{\char`\>}
\def\PYGZsh{\char`\#}
\def\PYGZpc{\char`\%}
\def\PYGZdl{\char`\$}
\def\PYGZhy{\char`\-}
\def\PYGZsq{\char`\'}
\def\PYGZdq{\char`\"}
\def\PYGZti{\char`\~}
% for compatibility with earlier versions
\def\PYGZat{@}
\def\PYGZlb{[}
\def\PYGZrb{]}
\makeatother

\renewcommand\PYGZsq{\textquotesingle}

\begin{document}

\maketitle
\tableofcontents
\phantomsection\label{index::doc}


Contents:


\chapter{Pyleoclim}
\label{Introduction::doc}\label{Introduction:welcome-to-pyleoclim-s-documentation}\label{Introduction:pyleoclim}

\section{What is it?}
\label{Introduction:what-is-it}
Pyleoclim is a Python package primarily geared towards the analysis and visualization of paleoclimate data.
Such data often come in the form of timeseries with missing values and age uncertainties, and the package
includes several low-level methods to deal with these issues, as well as high-level methods that re-use those
to perform scientific workflows.

The package assumes that the data are stored in the Linked Paleo Data (\href{http://www.clim-past.net/12/1093/2016/}{LiPD})
format and makes extensive use of the \href{http://nickmckay.github.io/LiPD-utilities/}{LiPD utilities}. The package
is aware of age ensembles stored via LiPD and uses them for time-uncertain analyses very much like \href{http://nickmckay.github.io/GeoChronR/}{GeoChronR}.

\textbf{Current Capabilities:}
\begin{itemize}
\item {} 
binning

\item {} 
interpolation

\item {} 
plotting maps, timeseries, and basic age model information

\end{itemize}

\textbf{Future capabilities:}
\begin{itemize}
\item {} 
paleo-aware correlation analysis (isopersistent, isospectral, and classical t-test)

\item {} 
paleo-aware singular spectrum analysis (AR(1) null eigenvalue identification, missing data)

\item {} 
spectral analysis (Multi-Taper Method, Lomb-Scargle)

\item {} 
weighted wavelet Z transform (WWZ)

\item {} 
cross-wavelet analysis

\item {} 
index reconstruction

\item {} 
climate reconstruction

\item {} 
ensemble methods for most of the above

\end{itemize}


\section{Version Information}
\label{Introduction:version-information}
\begin{DUlineblock}{0em}
\item[] 0.1.4: Rename functions using camel case convention and consistency with LiPD utilities version 0.1.8.5
\item[] 0.1.3: Compatible with LiPD utilities version 0.1.8.5
\item[]
\begin{DUlineblock}{\DUlineblockindent}
\item[] Function openLiPD() renamed openLiPDs()
\end{DUlineblock}
\item[] 0.1.2: Compatible with LiPD utilities version 0.1.8.3
\item[]
\begin{DUlineblock}{\DUlineblockindent}
\item[] Uses Basemap instead of cartopy
\end{DUlineblock}
\item[] 0.1.1: Freezes the package prior to version 0.1.8.2 of LiPD utilities
\item[] 0.1.0: First release
\end{DUlineblock}


\section{Installation}
\label{Introduction:installation}
Python v3.5+ is required
Pyleoclim is published through Pypi and easily installed via pip:

\begin{Verbatim}[commandchars=\\\{\}]
pip install pyleoclim
\end{Verbatim}


\section{Quickstart guide}
\label{Introduction:quickstart-guide}\begin{enumerate}
\item {} 
Open your command line application (Terminal or Command Prompt)

\item {} 
Install with command:

\end{enumerate}
\begin{quote}

pip install pyleoclim
\end{quote}
\begin{enumerate}
\setcounter{enumi}{2}
\item {} 
Wait for installation to complete, then:

\end{enumerate}
\begin{enumerate}
\item {} 
Import the package into your favorite Python environment (we recommend the use of Spyder, which comes standard with the Anaconda build)

\item {} 
Use Jupyter Notebook to go through the tutorial contained in the \href{https://github.com/LinkedEarth/Pyleoclim\_util/tree/master/Example}{PyleolimQuickstart.ipynb}

\end{enumerate}


\section{Requirements}
\label{Introduction:requirements}\begin{itemize}
\item {} 
LiPD v0.1.8.5

\item {} 
pandas v0.19+

\item {} 
numpy v1.12+

\item {} 
matplotlib v2.0+

\item {} 
basemap v1.0.7+

\end{itemize}

The installer will automatically check for the needed updates.


\section{Further information}
\label{Introduction:further-information}
\begin{DUlineblock}{0em}
\item[] GitHub: \href{https://github.com/LinkedEarth/Pyleoclim\_util}{https://github.com/LinkedEarth/Pyleoclim\_util}
\item[] LinkedEarth: \href{http://linked.earth}{http://linked.earth}
\item[] Python and Anaconda: \href{http://conda.pydata.org/docs/test-drive.html}{http://conda.pydata.org/docs/test-drive.html}
\item[] Jupyter Notebook: \href{http://jupyter.org/}{http://jupyter.org/}
\end{DUlineblock}


\section{Contact}
\label{Introduction:contact}
Please report issues to \href{mailto:linkedearth@gmail.com}{linkedearth@gmail.com}


\section{License}
\label{Introduction:license}
The project is licensed under the \href{https://github.com/LinkedEarth/Pyleoclim\_util/blob/master/license}{GNU Public License} .


\section{Disclaimer}
\label{Introduction:disclaimer}
This material is based upon work supported by the U.S. National Science Foundation under Grant Number
ICER-1541029. Any opinions, findings, and conclusions or recommendations expressed in this material are those
of the investigators and do not necessarily reflect the views of the National Science Foundation.


\chapter{Main Functions}
\label{Main::doc}\label{Main:main-functions}

\section{Getting started}
\label{Main:getting-started}
Pyleoclim relies heavily on the concept of timeseries objects introduced in
\href{http://www.clim-past.net/12/1093/2016/}{LiPD} and implemented in the
\href{http://nickmckay.github.io/LiPD-utilities/}{LiPD utilities}.

Briefly, timeseries objects are dictionaries containing the ChronData values and
PaleoData values as well as the metadata associated with the record. If one record
has three ProxyObservations (e.g., Mg/Ca, d18O, d13C) then it will have three timeseries
objects, one for each of the observations.

The LiPD utilities function lipd.extractTs() returns a list of dictionaries for
the selected LiPD files, which need to be passed to Pyleoclim along with the path
to the directory containing the LiPD files.

This is done through the function pyleoclim.openLiPDs:
\index{openLipds() (in module pyleoclim)}

\begin{fulllineitems}
\phantomsection\label{Main:pyleoclim.openLipds}\pysiglinewithargsret{\code{pyleoclim.}\bfcode{openLipds}}{\emph{path='`}, \emph{ts\_list='`}}{}
Load and extract timeseries objects from LiPD files.

Allows to load and extract timeseries objects into the workspace for use
with Pyleoclim. This can be done by the user previously, using the LiPD
utilities and passed into the function's argumenta. If no timeseries objects
are found by other functions, this function will be triggered automatically
without arguments.
\begin{quote}\begin{description}
\item[{Parameters}] \leavevmode\begin{itemize}
\item {} 
\textbf{\texttt{path}} (\emph{\texttt{string}}) -- the path to the LiPD file. If not specified, will
trigger the LiPD utilities GUI.

\item {} 
\textbf{\texttt{ts\_list}} (\emph{\texttt{list}}) -- the list of available timeseries objects
obtained from lipd.extractTs().

\end{itemize}

\end{description}\end{quote}

\begin{notice}{warning}{Warning:}
if specifying a list, path should also be specified.
\end{notice}
\paragraph{Examples}

\begin{Verbatim}[commandchars=\\\{\}]
\PYG{g+gp}{\PYGZgt{}\PYGZgt{}\PYGZgt{} }\PYG{n}{pyleoclim}\PYG{o}{.}\PYG{n}{openLipds}\PYG{p}{(}\PYG{n}{path} \PYG{o}{=} \PYG{l+s+s2}{\PYGZdq{}}\PYG{l+s+s2}{/Users/deborahkhider/Documents/LiPD}\PYG{l+s+s2}{\PYGZdq{}}\PYG{p}{)}
\PYG{g+go}{Found: 12 LiPD file(s)}
\PYG{g+go}{processing: Crystal.McCabe\PYGZhy{}Glynn.2013.lpd}
\PYG{g+go}{processing: MD01\PYGZhy{}2412.Harada.2006.lpd}
\PYG{g+go}{processing: MD98\PYGZhy{}2170.Stott.2004.lpd}
\PYG{g+go}{processing: MD982176.Stott.2004.lpd}
\PYG{g+go}{processing: O2kLR\PYGZhy{}EmeraldBasin.Sachs.2007.lpd}
\PYG{g+go}{processing: Ocean2kHR\PYGZhy{}AtlanticBahamasTOTORosenheim2005.lpd}
\PYG{g+go}{processing: Ocean2kHR\PYGZhy{}AtlanticCapeVerdeMoses2006.lpd}
\PYG{g+go}{processing: Ocean2kHR\PYGZhy{}AtlanticMontegoBayHaaseSchramm2003.lpd}
\PYG{g+go}{processing: Ocean2kHR\PYGZhy{}AtlanticPrincipeSwart1998.lpd}
\PYG{g+go}{processing: Ocean2kHR\PYGZhy{}PacificClippertonClipp2bWu2014.lpd}
\PYG{g+go}{processing: Ocean2kHR\PYGZhy{}PacificNauruGuilderson1999.lpd}
\PYG{g+go}{processing: ODP1098B.lpd}
\PYG{g+go}{extracting: ODP1098B.lpd}
\PYG{g+go}{extracting: MD98\PYGZhy{}2170.Stott.2004.lpd}
\PYG{g+go}{extracting: Ocean2kHR\PYGZhy{}PacificClippertonClipp2bWu2014.lpd}
\PYG{g+go}{extracting: Ocean2kHR\PYGZhy{}AtlanticBahamasTOTORosenheim2005.lpd}
\PYG{g+go}{extracting: Ocean2kHR\PYGZhy{}AtlanticPrincipeSwart1998.lpd}
\PYG{g+go}{extracting: Ocean2kHR\PYGZhy{}AtlanticMontegoBayHaaseSchramm2003.lpd}
\PYG{g+go}{extracting: MD982176.Stott.2004.lpd}
\PYG{g+go}{extracting: Ocean2kHR\PYGZhy{}PacificNauruGuilderson1999.lpd}
\PYG{g+go}{extracting: O2kLR\PYGZhy{}EmeraldBasin.Sachs.2007.lpd}
\PYG{g+go}{extracting: Crystal.McCabe\PYGZhy{}Glynn.2013.lpd}
\PYG{g+go}{extracting: Ocean2kHR\PYGZhy{}AtlanticCapeVerdeMoses2006.lpd}
\PYG{g+go}{extracting: MD01\PYGZhy{}2412.Harada.2006.lpd}
\PYG{g+go}{Finished time series: 31 objects}
\PYG{g+go}{Process Complete}
\end{Verbatim}

\end{fulllineitems}



\section{Mapping}
\label{Main:mapping}\index{mapAll() (in module pyleoclim)}

\begin{fulllineitems}
\phantomsection\label{Main:pyleoclim.mapAll}\pysiglinewithargsret{\code{pyleoclim.}\bfcode{mapAll}}{\emph{markersize=50}, \emph{saveFig=False}, \emph{dir='`}, \emph{format='eps'}}{}
Map all the available records loaded into the workspace by archiveType.
\begin{description}
\item[{Map of all the records into the workspace by archiveType.}] \leavevmode
Uses the default color palette. Enter pyleoclim.plot\_default for detail.

\end{description}
\begin{quote}\begin{description}
\item[{Parameters}] \leavevmode\begin{itemize}
\item {} 
\textbf{\texttt{markersize}} (\emph{\texttt{int}}) -- The size of the markers. Default is 50

\item {} 
\textbf{\texttt{saveFig}} (\emph{\texttt{bool}}) -- Default is to not save the figure

\item {} 
\textbf{\texttt{dir}} (\emph{\texttt{str}}) -- The absolute path of the directory in which to save the
figure. If not provided, creates a default folder called `figures'
in the LiPD working directory (lipd.path).

\item {} 
\textbf{\texttt{format}} (\emph{\texttt{str}}) -- One of the file extensions supported by the active
backend. Default is ``eps''. Most backend support png, pdf, ps, eps,
and svg.

\end{itemize}

\item[{Returns}] \leavevmode
The figure

\end{description}\end{quote}
\paragraph{Examples}

\begin{Verbatim}[commandchars=\\\{\}]
\PYG{g+gp}{\PYGZgt{}\PYGZgt{}\PYGZgt{} }\PYG{n}{fig} \PYG{o}{=} \PYG{n}{pyleoclim}\PYG{o}{.}\PYG{n}{mapAll}\PYG{p}{(}\PYG{p}{)}
\end{Verbatim}

\end{fulllineitems}

\index{mapLipd() (in module pyleoclim)}

\begin{fulllineitems}
\phantomsection\label{Main:pyleoclim.mapLipd}\pysiglinewithargsret{\code{pyleoclim.}\bfcode{mapLipd}}{\emph{name='`}, \emph{countries=True}, \emph{counties=False}, \emph{rivers=False}, \emph{states=False}, \emph{background='shadedrelief'}, \emph{scale=0.5}, \emph{markersize=50}, \emph{marker='default'}, \emph{saveFig=False}, \emph{dir='`}, \emph{format='eps'}}{}
Create a Map for a single record

Orthographic projection map of a single record.
\begin{quote}\begin{description}
\item[{Parameters}] \leavevmode\begin{itemize}
\item {} 
\textbf{\texttt{name}} (\emph{\texttt{str}}) -- the name of the LiPD file. \textbf{WITH THE .LPD EXTENSION!}.
If not provided, will prompt the user for one

\item {} 
\textbf{\texttt{countries}} (\emph{\texttt{bool}}) -- Draws the country borders. Default is on (True).

\item {} 
\textbf{\texttt{counties}} (\emph{\texttt{bool}}) -- Draws the USA counties. Default is off (False).

\item {} 
\textbf{\texttt{states}} (\emph{\texttt{bool}}) -- Draws the American and Australian states borders.
Default is off (False)

\item {} 
\textbf{\texttt{background}} (\emph{\texttt{str}}) -- Plots one of the following images on the map:
bluemarble, etopo, shadedrelief, or none (filled continents).
Default is shadedrelief

\item {} 
\textbf{\texttt{scale}} (\emph{\texttt{float}}) -- useful to downgrade the original image resolution to
speed up the process. Default is 0.5.

\item {} 
\textbf{\texttt{markersize}} (\emph{\texttt{int}}) -- default is 100

\item {} 
\textbf{\texttt{marker}} (\emph{\texttt{str}}) -- a string (or list) containing the color and shape of the
marker. Default is by archiveType. Type pyleo.plot\_default to see
the default palette.

\item {} 
\textbf{\texttt{saveFig}} (\emph{\texttt{bool}}) -- default is to not save the figure

\item {} 
\textbf{\texttt{dir}} (\emph{\texttt{str}}) -- the full path of the directory in which to save the figure.
If not provided, creates a default folder called `figures' in the
LiPD working directory (lipd.path).

\item {} 
\textbf{\texttt{format}} (\emph{\texttt{str}}) -- One of the file extensions supported by the active
backend. Default is ``eps''. Most backend support png, pdf, ps, eps,
and svg.

\end{itemize}

\item[{Returns}] \leavevmode
The figure

\end{description}\end{quote}
\paragraph{Examples}

\begin{Verbatim}[commandchars=\\\{\}]
\PYG{g+gp}{\PYGZgt{}\PYGZgt{}\PYGZgt{} }\PYG{n}{fig} \PYG{o}{=} \PYG{n}{pyleoclim}\PYG{o}{.}\PYG{n}{mapLipd}\PYG{p}{(}\PYG{n}{markersize}\PYG{o}{=}\PYG{l+m+mi}{100}\PYG{p}{)}
\end{Verbatim}

\end{fulllineitems}



\section{Plotting}
\label{Main:plotting}\index{plotTs() (in module pyleoclim)}

\begin{fulllineitems}
\phantomsection\label{Main:pyleoclim.plotTs}\pysiglinewithargsret{\code{pyleoclim.}\bfcode{plotTs}}{\emph{timeseries='`}, \emph{x\_axis='`}, \emph{markersize=50}, \emph{marker='default'}, \emph{saveFig=False}, \emph{dir='`}, \emph{format='eps'}}{}
Plot a single time series.
\begin{quote}\begin{description}
\item[{Parameters}] \leavevmode\begin{itemize}
\item {} 
\textbf{\texttt{timeseries}} (\emph{\texttt{A}}) -- By default, will prompt the user for one.

\item {} 
\textbf{\texttt{x\_axis}} (\emph{\texttt{str}}) -- The representation against which to plot the paleo-data.
Options are ``age'', ``year'', and ``depth''. Default is to let the
system choose if only one available or prompt the user.

\item {} 
\textbf{\texttt{markersize}} (\emph{\texttt{int}}) -- default is 50.

\item {} 
\textbf{\texttt{marker}} (\emph{\texttt{str}}) -- a string (or list) containing the color and shape of the
marker. Default is by archiveType. Type pyleo.plot\_default to see
the default palette.

\item {} 
\textbf{\texttt{saveFig}} (\emph{\texttt{bool}}) -- default is to not save the figure

\item {} 
\textbf{\texttt{dir}} (\emph{\texttt{str}}) -- the full path of the directory in which to save the figure.
If not provided, creates a default folder called `figures' in the
LiPD working directory (lipd.path).

\item {} 
\textbf{\texttt{format}} (\emph{\texttt{str}}) -- One of the file extensions supported by the active
backend. Default is ``eps''. Most backend support png, pdf, ps, eps,
and svg.

\end{itemize}

\item[{Returns}] \leavevmode
The figure.

\end{description}\end{quote}
\paragraph{Examples}

\begin{Verbatim}[commandchars=\\\{\}]
\PYG{g+gp}{\PYGZgt{}\PYGZgt{}\PYGZgt{} }\PYG{n}{fig} \PYG{o}{=} \PYG{n}{pyleoclim}\PYG{o}{.}\PYG{n}{plotTs}\PYG{p}{(}\PYG{n}{marker} \PYG{o}{=} \PYG{l+s+s1}{\PYGZsq{}}\PYG{l+s+s1}{rs}\PYG{l+s+s1}{\PYGZsq{}}\PYG{p}{)}
\end{Verbatim}

\end{fulllineitems}



\subsection{Summary Plots}
\label{Main:summary-plots}
Summary plots are a special categories of plots enabled by Pyleoclim.
They allow to plot specific information about a timeseries but are not customizable.
\index{basicSummary() (in module pyleoclim)}

\begin{fulllineitems}
\phantomsection\label{Main:pyleoclim.basicSummary}\pysiglinewithargsret{\code{pyleoclim.}\bfcode{basicSummary}}{\emph{timeseries='`}, \emph{x\_axis='`}, \emph{saveFig=False}, \emph{format='eps'}, \emph{dir='`}}{}
Makes a basic summary plot

Plots the following information: the time series, location map,
Age-Depth profile if both are available from the paleodata, Metadata
\paragraph{Notes}

The plots use default settings from the MapLiPD and plotTS methods.
\begin{quote}\begin{description}
\item[{Parameters}] \leavevmode\begin{itemize}
\item {} 
\textbf{\texttt{timeseries}} -- By default, will prompt for one.

\item {} 
\textbf{\texttt{x-axis}} (\emph{\texttt{str}}) -- The representation against which to plot the paleo-data.
Options are ``age'', ``year'', and ``depth''. Default is to let the
system choose if only one available or prompt the user.

\item {} 
\textbf{\texttt{saveFig}} (\emph{\texttt{bool}}) -- default is to not save the figure

\item {} 
\textbf{\texttt{dir}} (\emph{\texttt{str}}) -- the full path of the directory in which to save the figure.
If not provided, creates a default folder called `figures' in the
LiPD working directory (lipd.path).

\item {} 
\textbf{\texttt{format}} (\emph{\texttt{str}}) -- One of the file extensions supported by the active
backend. Default is ``eps''. Most backend support png, pdf, ps, eps,
and svg.

\end{itemize}

\item[{Returns}] \leavevmode
The figure.

\end{description}\end{quote}
\paragraph{Examples}

\begin{Verbatim}[commandchars=\\\{\}]
\PYG{g+gp}{\PYGZgt{}\PYGZgt{}\PYGZgt{} }\PYG{n}{fig} \PYG{o}{=} \PYG{n}{pyleoclim}\PYG{o}{.}\PYG{n}{basicSummary}\PYG{p}{(}\PYG{p}{)}
\end{Verbatim}

\end{fulllineitems}



\section{Statistics}
\label{Main:statistics}\index{statsTs() (in module pyleoclim)}

\begin{fulllineitems}
\phantomsection\label{Main:pyleoclim.statsTs}\pysiglinewithargsret{\code{pyleoclim.}\bfcode{statsTs}}{\emph{timeseries='`}}{}
Calculate the mean and standard deviation of a timeseries
\begin{quote}\begin{description}
\item[{Parameters}] \leavevmode
\textbf{\texttt{timeseries}} -- sytem will prompt for one if not given

\item[{Returns}] \leavevmode
The mean and standard deviation

\end{description}\end{quote}
\paragraph{Examples}

\begin{Verbatim}[commandchars=\\\{\}]
\PYG{g+gp}{\PYGZgt{}\PYGZgt{}\PYGZgt{} }\PYG{n}{mean}\PYG{p}{,}\PYG{n}{std} \PYG{o}{=} \PYG{n}{pyleoclim}\PYG{o}{.}\PYG{n}{statsTs}\PYG{p}{(}\PYG{p}{)}
\PYG{g+go}{0 :  Ocean2kHR\PYGZhy{}AtlanticMontegoBayHaaseSchramm2003 :  Sr\PYGZus{}Ca}
\PYG{g+go}{1 :  Ocean2kHR\PYGZhy{}AtlanticMontegoBayHaaseSchramm2003 :  d18O}
\PYG{g+go}{2 :  O2kLR\PYGZhy{}EmeraldBasin.Sachs.2007 :  notes}
\PYG{g+go}{3 :  O2kLR\PYGZhy{}EmeraldBasin.Sachs.2007 :  temperature}
\PYG{g+go}{4 :  O2kLR\PYGZhy{}EmeraldBasin.Sachs.2007 :  Uk37}
\PYG{g+go}{5 :  O2kLR\PYGZhy{}EmeraldBasin.Sachs.2007 :  notes}
\PYG{g+go}{6 :  O2kLR\PYGZhy{}EmeraldBasin.Sachs.2007 :  temperature}
\PYG{g+go}{7 :  O2kLR\PYGZhy{}EmeraldBasin.Sachs.2007 :  Uk37}
\PYG{g+go}{8 :  ODP1098B :  SST}
\PYG{g+go}{9 :  ODP1098B :  TEX86}
\PYG{g+go}{10 :  MD01\PYGZhy{}2412.Harada.2006 :  calyrbp}
\PYG{g+go}{11 :  MD01\PYGZhy{}2412.Harada.2006 :  sst}
\PYG{g+go}{12 :  MD01\PYGZhy{}2412.Harada.2006 :  uk37}
\PYG{g+go}{13 :  Crystal.McCabe\PYGZhy{}Glynn.2013 :  s180carbVPDB}
\PYG{g+go}{14 :  Crystal.McCabe\PYGZhy{}Glynn.2013 :  sst.anom}
\PYG{g+go}{15 :  Ocean2kHR\PYGZhy{}AtlanticCapeVerdeMoses2006 :  d18O}
\PYG{g+go}{16 :  Ocean2kHR\PYGZhy{}PacificNauruGuilderson1999 :  d13C}
\PYG{g+go}{17 :  Ocean2kHR\PYGZhy{}PacificNauruGuilderson1999 :  d18O}
\PYG{g+go}{18 :  Ocean2kHR\PYGZhy{}AtlanticBahamasTOTORosenheim2005 :  d18O}
\PYG{g+go}{19 :  Ocean2kHR\PYGZhy{}AtlanticBahamasTOTORosenheim2005 :  Sr\PYGZus{}Ca}
\PYG{g+go}{20 :  Ocean2kHR\PYGZhy{}AtlanticPrincipeSwart1998 :  d13C}
\PYG{g+go}{21 :  Ocean2kHR\PYGZhy{}AtlanticPrincipeSwart1998 :  d18O}
\PYG{g+go}{22 :  MD98\PYGZhy{}2170.Stott.2004 :  d18o}
\PYG{g+go}{23 :  MD98\PYGZhy{}2170.Stott.2004 :  RMSE}
\PYG{g+go}{24 :  MD98\PYGZhy{}2170.Stott.2004 :  mg}
\PYG{g+go}{25 :  MD98\PYGZhy{}2170.Stott.2004 :  d18ow}
\PYG{g+go}{26 :  Ocean2kHR\PYGZhy{}PacificClippertonClipp2bWu2014 :  Sr\PYGZus{}Ca}
\PYG{g+go}{27 :  MD982176.Stott.2004 :  Mg/Ca\PYGZhy{}g.rub}
\PYG{g+go}{28 :  MD982176.Stott.2004 :  sst}
\PYG{g+go}{29 :  MD982176.Stott.2004 :  d18Ob.rub}
\PYG{g+go}{30 :  MD982176.Stott.2004 :  d18Ow\PYGZhy{}s}
\PYG{g+go}{Enter the number of the variable you wish to use: 12}
\end{Verbatim}

\begin{Verbatim}[commandchars=\\\{\}]
\PYG{g+gp}{\PYGZgt{}\PYGZgt{}\PYGZgt{} }\PYG{k}{print}\PYG{p}{(}\PYG{n}{mean}\PYG{p}{)}
\PYG{g+go}{0.401759365994}
\end{Verbatim}

\begin{Verbatim}[commandchars=\\\{\}]
\PYG{g+gp}{\PYGZgt{}\PYGZgt{}\PYGZgt{} }\PYG{k}{print}\PYG{p}{(}\PYG{n}{std}\PYG{p}{)}
\PYG{g+go}{0.0821452359532}
\end{Verbatim}

\end{fulllineitems}

\index{binTs() (in module pyleoclim)}

\begin{fulllineitems}
\phantomsection\label{Main:pyleoclim.binTs}\pysiglinewithargsret{\code{pyleoclim.}\bfcode{binTs}}{\emph{timeseries='`}, \emph{x\_axis='`}, \emph{bin\_size='`}, \emph{start='`}, \emph{end='`}}{}
Bin the paleoData values of the timeseries
\begin{quote}\begin{description}
\item[{Parameters}] \leavevmode\begin{itemize}
\item {} 
\textbf{\texttt{By default, will prompt the user for one.}} (\emph{\texttt{timeseries.}}) -- 

\item {} 
\textbf{\texttt{x-axis}} (\emph{\texttt{str}}) -- The representation against which to plot the paleo-data.
Options are ``age'', ``year'', and ``depth''. Default is to let the
system  choose if only one available or prompt the user.

\item {} 
\textbf{\texttt{bin\_size}} (\emph{\texttt{float}}) -- the size of the bins to be used. By default,
will prompt for one

\item {} 
\textbf{\texttt{start}} (\emph{\texttt{float}}) -- Start time/age/depth. Default is the minimum

\item {} 
\textbf{\texttt{end}} (\emph{\texttt{float}}) -- End time/age/depth. Default is the maximum

\end{itemize}

\item[{Returns}] \leavevmode

binned\_data- the binned output,

bins-  the bins (centered on the median, i.e. the 100-200 bin is 150),

n-  number of data points in each bin,

error- the standard error on the mean in each bin


\end{description}\end{quote}
\paragraph{Example}

\begin{Verbatim}[commandchars=\\\{\}]
\PYG{g+gp}{\PYGZgt{}\PYGZgt{}\PYGZgt{} }\PYG{n}{ts} \PYG{o}{=} \PYG{n}{pyleoclim}\PYG{o}{.}\PYG{n}{timeseries\PYGZus{}list}\PYG{p}{[}\PYG{l+m+mi}{28}\PYG{p}{]}
\PYG{g+gp}{\PYGZgt{}\PYGZgt{}\PYGZgt{} }\PYG{n}{bin\PYGZus{}size} \PYG{o}{=} \PYG{l+m+mi}{200}
\PYG{g+gp}{\PYGZgt{}\PYGZgt{}\PYGZgt{} }\PYG{n}{bins}\PYG{p}{,} \PYG{n}{binned\PYGZus{}data}\PYG{p}{,} \PYG{n}{n}\PYG{p}{,} \PYG{n}{error} \PYG{o}{=} \PYG{n}{pyleoclim}\PYG{o}{.}\PYG{n}{binTs}\PYG{p}{(}\PYG{n}{timeseries} \PYG{o}{=} \PYG{n}{ts}\PYG{p}{,} \PYG{n}{bin\PYGZus{}size} \PYG{o}{=} \PYG{n}{bin\PYGZus{}size}\PYG{p}{)}
\PYG{g+go}{Do you want to plot vs time or depth?}
\PYG{g+go}{Enter 0 for time and 1 for depth: 0}
\end{Verbatim}

\begin{Verbatim}[commandchars=\\\{\}]
\PYG{g+gp}{\PYGZgt{}\PYGZgt{}\PYGZgt{} }\PYG{k}{print}\PYG{p}{(}\PYG{n}{bins}\PYG{p}{)}
\PYG{g+go}{[   239.3    439.3    639.3 ...,  14439.3  14639.3  14839.3]}
\end{Verbatim}

\begin{Verbatim}[commandchars=\\\{\}]
\PYG{g+gp}{\PYGZgt{}\PYGZgt{}\PYGZgt{} }\PYG{k}{print}\PYG{p}{(}\PYG{n}{binned\PYGZus{}data}\PYG{p}{)}
\PYG{g+go}{[28.440000000000005, 28.920000000000005, 28.657142857142862,}
\PYG{g+go}{28.939999999999998, 28.733333333333334, 28.949999999999999, 28.75,}
\PYG{g+go}{28.899999999999999, 28.75, 28.566666666666663, 28.800000000000001,}
\PYG{g+go}{29.049999999999997, 29.233333333333334, 29.274999999999999,}
\PYG{g+go}{29.057142857142857, 28.699999999999999, 29.433333333333334,}
\PYG{g+go}{28.575000000000003, 28.733333333333331, 28.48, 28.733333333333331,}
\PYG{g+go}{28.766666666666666, 29.166666666666668, 29.18, 29.600000000000001,}
\PYG{g+go}{29.300000000000001, 28.949999999999999, 29.475000000000001,}
\PYG{g+go}{29.333333333333332, 29.800000000000001, 29.016666666666666,}
\PYG{g+go}{29.349999999999998, 29.485714285714288, 28.850000000000001,}
\PYG{g+go}{29.366666666666664, 28.699999999999999, 29.233333333333334,}
\PYG{g+go}{29.366666666666664, 29.5, 29.350000000000001, 29.699999999999999,}
\PYG{g+go}{29.300000000000001, 29.233333333333334, 29.300000000000001,}
\PYG{g+go}{29.300000000000001, 29.600000000000001, 28.950000000000003,}
\PYG{g+go}{29.166666666666668, 28.799999999999997, 28.975000000000001,}
\PYG{g+go}{29.033333333333331, 28.649999999999999, 28.450000000000003,}
\PYG{g+go}{28.533333333333331, 28.599999999999998, 28.25, 28.0,}
\PYG{g+go}{28.550000000000001, 28.799999999999997, 28.350000000000001,}
\PYG{g+go}{27.699999999999999, 27.149999999999999, 27.666666666666668,}
\PYG{g+go}{26.800000000000001, 26.700000000000003, 26.800000000000001,}
\PYG{g+go}{26.5, 26.850000000000001, 26.5, 26.5, 26.0, 26.899999999999999,}
\PYG{g+go}{26.5, 26.100000000000001]}
\end{Verbatim}

\end{fulllineitems}

\index{interpTs() (in module pyleoclim)}

\begin{fulllineitems}
\phantomsection\label{Main:pyleoclim.interpTs}\pysiglinewithargsret{\code{pyleoclim.}\bfcode{interpTs}}{\emph{timeseries='`}, \emph{x\_axis='`}, \emph{interp\_step='`}, \emph{start='`}, \emph{end='`}}{}
Simple linear interpolation

Simple linear interpolation of the data using the numpy.interp method
\begin{quote}\begin{description}
\item[{Parameters}] \leavevmode\begin{itemize}
\item {} 
\textbf{\texttt{Default is blank, will prompt for it}} (\emph{\texttt{timeseries.}}) -- 

\item {} 
\textbf{\texttt{x-axis}} (\emph{\texttt{str}}) -- The representation against which to plot the paleo-data.
Options are ``age'', ``year'', and ``depth''. Default is to let the
system choose if only one available or prompt the user.

\item {} 
\textbf{\texttt{interp\_step}} (\emph{\texttt{float}}) -- the step size. By default, will prompt the user.

\item {} 
\textbf{\texttt{start}} (\emph{\texttt{float}}) -- Start time/age/depth. Default is the minimum

\item {} 
\textbf{\texttt{end}} (\emph{\texttt{float}}) -- End time/age/depth. Default is the maximum

\end{itemize}

\item[{Returns}] \leavevmode

interp\_age - the interpolated age/year/depth according to the end/start
and time step,

interp\_values - the interpolated values


\end{description}\end{quote}
\paragraph{Examples}

\begin{Verbatim}[commandchars=\\\{\}]
\PYG{g+gp}{\PYGZgt{}\PYGZgt{}\PYGZgt{} }\PYG{n}{ts} \PYG{o}{=} \PYG{n}{pyleoclim}\PYG{o}{.}\PYG{n}{timeseries\PYGZus{}list}\PYG{p}{[}\PYG{l+m+mi}{28}\PYG{p}{]}
\PYG{g+gp}{\PYGZgt{}\PYGZgt{}\PYGZgt{} }\PYG{n}{interp\PYGZus{}step} \PYG{o}{=} \PYG{l+m+mi}{200}
\PYG{g+gp}{\PYGZgt{}\PYGZgt{}\PYGZgt{} }\PYG{n}{interp\PYGZus{}age}\PYG{p}{,} \PYG{n}{interp\PYGZus{}values} \PYG{o}{=} \PYG{n}{pyleoclim}\PYG{o}{.}\PYG{n}{interpTs}\PYG{p}{(}\PYG{n}{timeseries} \PYG{o}{=} \PYG{n}{ts}\PYG{p}{,} \PYG{n}{interp\PYGZus{}step} \PYG{o}{=} \PYG{n}{interp\PYGZus{}step}\PYG{p}{)}
\PYG{g+go}{Do you want to plot vs time or depth?}
\PYG{g+go}{Enter 0 for time and 1 for depth: 0}
\end{Verbatim}

\begin{Verbatim}[commandchars=\\\{\}]
\PYG{g+gp}{\PYGZgt{}\PYGZgt{}\PYGZgt{} }\PYG{k}{print}\PYG{p}{(}\PYG{n}{interp\PYGZus{}age}\PYG{p}{)}
\PYG{g+go}{[   139.3    339.3    539.3 ...,  14339.3  14539.3  14739.3]}
\end{Verbatim}

\begin{Verbatim}[commandchars=\\\{\}]
\PYG{g+gp}{\PYGZgt{}\PYGZgt{}\PYGZgt{} }\PYG{k}{print}\PYG{p}{(}\PYG{n}{interp\PYGZus{}values}\PYG{p}{)}
\PYG{g+go}{[ 0.188       0.05981567 \PYGZhy{}0.04020261 ...,  1.20834663  1.47751854}
\PYG{g+go}{ 1.16054494]}
\end{Verbatim}

\end{fulllineitems}



\chapter{Basic Module}
\label{Basic::doc}\label{Basic:basic-module}
This module contains methods for basic manipulation of the Paleo/Chron Data.
\index{Basic (class in pyleoclim)}

\begin{fulllineitems}
\phantomsection\label{Basic:pyleoclim.Basic}\pysiglinewithargsret{\strong{class }\code{pyleoclim.}\bfcode{Basic}}{\emph{timeseries\_list}}{}
Basic manipulation of timeseries for scientific purpose.

Calculates statistics of timeseries, bin or interpolate data
\index{bin\_Ts() (pyleoclim.Basic static method)}

\begin{fulllineitems}
\phantomsection\label{Basic:pyleoclim.Basic.bin_Ts}\pysiglinewithargsret{\strong{static }\bfcode{bin\_Ts}}{\emph{timeseries}, \emph{x\_axis='`}, \emph{bin\_size='`}, \emph{start='`}, \emph{end='`}}{}
Bin the PaleoData values
\begin{quote}\begin{description}
\item[{Parameters}] \leavevmode\begin{itemize}
\item {} 
\textbf{\texttt{timeseries}} -- a single timeseries object. Use getTSO() to get one.

\item {} 
\textbf{\texttt{x-axis}} (\emph{\texttt{str}}) -- The representation against which to plot the
paleo-data. Options are ``age'', ``year'', and ``depth''. Default
is to let the system choose if only one available or prompt
the user.

\item {} 
\textbf{\texttt{bin\_size}} (\emph{\texttt{float}}) -- the size of the bins to be used.
By default, will prompt for one

\item {} 
\textbf{\texttt{start}} (\emph{\texttt{float}}) -- Start time/age/depth. Default is the minimum

\item {} 
\textbf{\texttt{end}} (\emph{\texttt{float}}) -- End time/age/depth. Default is the maximum

\end{itemize}

\end{description}\end{quote}
\begin{description}
\item[{Returnss:}] \leavevmode
binned\_data - the binned output

bins - the bins (centered on the median, i.e. the 100-200 bin is 150)

n - number of data points in each bin

error - the standard error on the mean in each bin

\end{description}

\end{fulllineitems}

\index{getValues() (pyleoclim.Basic static method)}

\begin{fulllineitems}
\phantomsection\label{Basic:pyleoclim.Basic.getValues}\pysiglinewithargsret{\strong{static }\bfcode{getValues}}{\emph{timeseries}}{}
Get the paleoData values from the timeseries object
\begin{quote}\begin{description}
\item[{Parameters}] \leavevmode
\textbf{\texttt{timeseries}} -- a single timeseries object. Use getTSO() to get
one from the dictionary

\end{description}\end{quote}

\end{fulllineitems}

\index{interp\_Ts() (pyleoclim.Basic static method)}

\begin{fulllineitems}
\phantomsection\label{Basic:pyleoclim.Basic.interp_Ts}\pysiglinewithargsret{\strong{static }\bfcode{interp\_Ts}}{\emph{timeseries}, \emph{x\_axis='`}, \emph{interp\_step='`}, \emph{start='`}, \emph{end='`}}{}
Linear interpolation of the PaleoData values
\begin{quote}\begin{description}
\item[{Parameters}] \leavevmode\begin{itemize}
\item {} 
\textbf{\texttt{timeseries}} -- a timeseries object. Default is blank, will prompt for it

\item {} 
\textbf{\texttt{x-axis}} (\emph{\texttt{str}}) -- The representation against which to plot the
paleo-data. Options are ``age'', ``year'', and ``depth''.
Default is to let the system choose if only one available
or prompt the user.

\item {} 
\textbf{\texttt{interp\_step}} (\emph{\texttt{float}}) -- the step size. By default, will prompt the user.

\item {} 
\textbf{\texttt{start}} (\emph{\texttt{float}}) -- Start time/age/depth. Default is the minimum

\item {} 
\textbf{\texttt{end}} (\emph{\texttt{float}}) -- End time/age/depth. Default is the maximum

\end{itemize}

\item[{Returns}] \leavevmode

interp\_age - the interpolated age/year/depth according to
the end/start and time step

interp\_values - the interpolated values


\end{description}\end{quote}

\end{fulllineitems}

\index{simpleStats() (pyleoclim.Basic method)}

\begin{fulllineitems}
\phantomsection\label{Basic:pyleoclim.Basic.simpleStats}\pysiglinewithargsret{\bfcode{simpleStats}}{\emph{timeseries='`}}{}
Compute the mean and standard deviation of a time series
\begin{quote}\begin{description}
\item[{Parameters}] \leavevmode
\textbf{\texttt{timeseries}} -- a single timeseries. Will prompt for one
if not available

\end{description}\end{quote}

\end{fulllineitems}


\end{fulllineitems}



\chapter{Mapping Module}
\label{Mapping::doc}\label{Mapping:mapping-module}\index{Map (class in pyleoclim)}

\begin{fulllineitems}
\phantomsection\label{Mapping:pyleoclim.Map}\pysiglinewithargsret{\strong{class }\code{pyleoclim.}\bfcode{Map}}{\emph{plot\_default}}{}
Create Maps using Basemap.

Uses the default color palette: pyleoclim.plot\_default
\index{map\_Lipd() (pyleoclim.Map method)}

\begin{fulllineitems}
\phantomsection\label{Mapping:pyleoclim.Map.map_Lipd}\pysiglinewithargsret{\bfcode{map\_Lipd}}{\emph{name='`}, \emph{countries=True}, \emph{counties=False}, \emph{rivers=False}, \emph{states=False}, \emph{background='shadedrelief'}, \emph{scale=0.5}, \emph{markersize=50}, \emph{marker='default'}, \emph{saveFig=False}, \emph{dir='`}, \emph{format='eps'}}{}
Makes a map for a single record.
\begin{quote}\begin{description}
\item[{Parameters}] \leavevmode\begin{itemize}
\item {} 
\textbf{\texttt{name}} (\emph{\texttt{str}}) -- the name of the LiPD file. \textbf{WITH THE
.LPD EXTENSION!}.If not provided, will prompt the user for one.

\item {} 
\textbf{\texttt{countries}} (\emph{\texttt{bool}}) -- Draws the country borders. Default is on (True).

\item {} 
\textbf{\texttt{counties}} (\emph{\texttt{bool}}) -- Draws the USA counties. Default is off (False).

\item {} 
\textbf{\texttt{states}} (\emph{\texttt{bool}}) -- Draws the American and Australian states borders.
Default is off (False)

\item {} 
\textbf{\texttt{background}} (\emph{\texttt{str}}) -- Plots one of the following images on the map:
bluemarble, etopo, shadedrelief, or none (filled continents).
Default is shadedrelief

\item {} 
\textbf{\texttt{scale}} (\emph{\texttt{float}}) -- useful to downgrade the original image resolution
to speed up the process. Default is 0.5.

\item {} 
\textbf{\texttt{markersize}} (\emph{\texttt{int}}) -- default is 100

\item {} 
\textbf{\texttt{marker}} (\emph{\texttt{str}}) -- a string (or list) containing the color and shape of
the marker. Default is by archiveType. Type pyleo.plot\_default
to see the default palette.

\item {} 
\textbf{\texttt{saveFig}} (\emph{\texttt{bool}}) -- default is to not save the figure

\item {} 
\textbf{\texttt{dir}} (\emph{\texttt{str}}) -- the full path of the directory in which to save the
figure. If not provided, creates a default folder called
`figures' in the LiPD working directory (lipd.path).

\item {} 
\textbf{\texttt{format}} (\emph{\texttt{str}}) -- One of the file extensions supported by the active
backend. Default is ``eps''. Most backend support png, pdf, ps,
eps, and svg.

\end{itemize}

\item[{Returns}] \leavevmode
The Figure

\end{description}\end{quote}

\end{fulllineitems}

\index{map\_all() (pyleoclim.Map method)}

\begin{fulllineitems}
\phantomsection\label{Mapping:pyleoclim.Map.map_all}\pysiglinewithargsret{\bfcode{map\_all}}{\emph{markersize=50}, \emph{saveFig=False}, \emph{dir='`}, \emph{format='eps'}}{}
Map all the available records loaded into the LiPD working directory by archiveType.
\begin{quote}\begin{description}
\item[{Parameters}] \leavevmode\begin{itemize}
\item {} 
\textbf{\texttt{markersize}} (\emph{\texttt{int}}) -- default is 50

\item {} 
\textbf{\texttt{saveFig}} (\emph{\texttt{bool}}) -- default is to save the figure

\item {} 
\textbf{\texttt{dir}} (\emph{\texttt{str}}) -- the full path of the directory in which to save the
figure. If not provided, creates a default folder called
`figures' in the LiPD working directory (lipd.path).

\item {} 
\textbf{\texttt{format}} (\emph{\texttt{str}}) -- One of the file extensions supported by the active
backend. Default is ``eps''. Most backend support png, pdf, ps,
eps, and svg.

\end{itemize}

\item[{Returns}] \leavevmode
The figure

\end{description}\end{quote}

\end{fulllineitems}


\end{fulllineitems}



\chapter{Timeseries Plots Module}
\label{TSPlots::doc}\label{TSPlots:timeseries-plots-module}\index{Plot (class in pyleoclim)}

\begin{fulllineitems}
\phantomsection\label{TSPlots:pyleoclim.Plot}\pysiglinewithargsret{\strong{class }\code{pyleoclim.}\bfcode{Plot}}{\emph{plot\_default}, \emph{timeseries\_list}}{}
Plot a timeseries
\index{plot\_Ts() (pyleoclim.Plot method)}

\begin{fulllineitems}
\phantomsection\label{TSPlots:pyleoclim.Plot.plot_Ts}\pysiglinewithargsret{\bfcode{plot\_Ts}}{\emph{timeseries='`}, \emph{x\_axis='`}, \emph{markersize=50}, \emph{marker='default'}, \emph{saveFig=False}, \emph{dir='`}, \emph{format='eps'}}{}
Plot a timeseries object
\begin{quote}\begin{description}
\item[{Parameters}] \leavevmode\begin{itemize}
\item {} 
\textbf{\texttt{timeseries}} -- A timeseries. By default, will prompt the user for one.

\item {} 
\textbf{\texttt{x\_axis}} (\emph{\texttt{str}}) -- The representation against which to plot the
paleo-data. Options are ``age'', ``year'', and ``depth''.
Default is to let the system choose if only one available or
prompt the user.

\item {} 
\textbf{\texttt{markersize}} (\emph{\texttt{int}}) -- default is 50.

\item {} 
\textbf{\texttt{marker}} (\emph{\texttt{str}}) -- a string (or list) containing the color and shape of
the marker. Default is by archiveType.Type pyleo.plot\_default
to see the default palette.

\item {} 
\textbf{\texttt{saveFig}} (\emph{\texttt{bool}}) -- default is to not save the figure

\item {} 
\textbf{\texttt{dir}} (\emph{\texttt{str}}) -- the full path of the directory in which to save the
figure. If not provided, creates a default folder called
`figures' in the LiPD working directory (lipd.path).

\item {} 
\textbf{\texttt{format}} (\emph{\texttt{str}}) -- One of the file extensions supported by the active
backend. Default is ``eps''. Most backend support png, pdf, ps, eps,
and svg.

\end{itemize}

\item[{Returns}] \leavevmode
The figure

\end{description}\end{quote}

\end{fulllineitems}

\index{plot\_agemodel() (pyleoclim.Plot method)}

\begin{fulllineitems}
\phantomsection\label{TSPlots:pyleoclim.Plot.plot_agemodel}\pysiglinewithargsret{\bfcode{plot\_agemodel}}{\emph{timeseries='`}, \emph{markersize=50}, \emph{marker='default'}, \emph{saveFig=True}, \emph{dir='`}, \emph{format='eps'}}{}
Make a simple age-depth profile
\begin{quote}\begin{description}
\item[{Parameters}] \leavevmode\begin{itemize}
\item {} 
\textbf{\texttt{timeseries}} -- A timeseries. By default, will prompt the user for one.

\item {} 
\textbf{\texttt{markersize}} (\emph{\texttt{int}}) -- default is 50.

\item {} 
\textbf{\texttt{marker}} (\emph{\texttt{str}}) -- a string (or list) containing the color and shape of
the marker. Default is by archiveType. Type pyleo.plot\_default
to see the default palette.

\item {} 
\textbf{\texttt{saveFig}} (\emph{\texttt{bool}}) -- default is to not save the figure

\item {} 
\textbf{\texttt{dir}} (\emph{\texttt{str}}) -- the full path of the directory in which to save the figure.
If not provided, creates a default folder called `figures' in
the LiPD working directory (lipd.path).

\item {} 
\textbf{\texttt{format}} (\emph{\texttt{str}}) -- One of the file extensions supported by the active
backend. Default is ``eps''. Most backend support png, pdf, ps,
eps, and svg.

\end{itemize}

\item[{Returns}] \leavevmode
The figure

\end{description}\end{quote}

\end{fulllineitems}


\end{fulllineitems}



\chapter{Summary Plots Module}
\label{SummaryPlots::doc}\label{SummaryPlots:summary-plots-module}\index{SummaryPlots (class in pyleoclim)}

\begin{fulllineitems}
\phantomsection\label{SummaryPlots:pyleoclim.SummaryPlots}\pysiglinewithargsret{\strong{class }\code{pyleoclim.}\bfcode{SummaryPlots}}{\emph{timeseries\_list}, \emph{plot\_default}}{}
Plots various summary figures for a LiPD record
\index{TsData() (pyleoclim.SummaryPlots method)}

\begin{fulllineitems}
\phantomsection\label{SummaryPlots:pyleoclim.SummaryPlots.TsData}\pysiglinewithargsret{\bfcode{TsData}}{\emph{timeseries='`}, \emph{x\_axis='`}}{}
Get the PaleoData with age/depth information

Get the necessary information for the TS plots/necessary to allow for
axes specification
\begin{quote}\begin{description}
\item[{Parameters}] \leavevmode\begin{itemize}
\item {} 
\textbf{\texttt{timeseries}} -- a single timeseries object.
By default, will prompt the user

\item {} 
\textbf{\texttt{x-axis}} (\emph{\texttt{str}}) -- The representation against which to plot the
paleo-data. Options are ``age'', ``year'', and ``depth''.
Default is to let the system choose if only one available
or prompt the user.

\end{itemize}

\item[{Returns}] \leavevmode

dataframe - a dataframe containg the x- and y-values

archiveType - the archiveType (for plot settings)

x\_axis\_label - the label for the x-axis

y\_axis\_label - the label for the y-axis

headers - the headers of the dataframe


\end{description}\end{quote}

\end{fulllineitems}

\index{agemodelData() (pyleoclim.SummaryPlots method)}

\begin{fulllineitems}
\phantomsection\label{SummaryPlots:pyleoclim.SummaryPlots.agemodelData}\pysiglinewithargsret{\bfcode{agemodelData}}{\emph{timeseries='`}}{}
Get the necessary information for the agemodel plot
\begin{quote}\begin{description}
\item[{Parameters}] \leavevmode
\textbf{\texttt{timeseries}} -- a single timeseries object. By default, will
prompt the user

\item[{Returns}] \leavevmode

depth - the depth values

age - the age values

x\_axis\_label - the label for the x-axis

y\_axis\_label - the label for the y-axis

archiveType - the archiveType (for default plot settings)


\end{description}\end{quote}

\end{fulllineitems}

\index{basic() (pyleoclim.SummaryPlots method)}

\begin{fulllineitems}
\phantomsection\label{SummaryPlots:pyleoclim.SummaryPlots.basic}\pysiglinewithargsret{\bfcode{basic}}{\emph{x\_axis='`}, \emph{timeseries='`}, \emph{saveFig=False}, \emph{format='eps'}, \emph{dir='`}}{}
Makes a basic summary plot

Plots the following information: the time series, location map,
Age-Depth profile if both are available from the paleodata, Metadata
\paragraph{Notes}

The plots use default settings from the MapLiPD and plotTS methods.
\begin{quote}\begin{description}
\item[{Parameters}] \leavevmode\begin{itemize}
\item {} 
\textbf{\texttt{new\_timeseries}} -- By default, will prompt for one.

\item {} 
\textbf{\texttt{x-axis}} (\emph{\texttt{str}}) -- The representation against which to plot the paleo-data.
Options are ``age'', ``year'', and ``depth''. Default is to let the
system choose if only one available or prompt the user.

\item {} 
\textbf{\texttt{saveFig}} (\emph{\texttt{bool}}) -- default is to not save the figure

\item {} 
\textbf{\texttt{dir}} (\emph{\texttt{str}}) -- the full path of the directory in which to save the figure.
If not provided, creates a default folder called `figures' in the
LiPD working directory (lipd.path).

\item {} 
\textbf{\texttt{format}} (\emph{\texttt{str}}) -- One of the file extensions supported by the active
backend. Default is ``eps''. Most backend support png, pdf, ps, eps,
and svg.

\end{itemize}

\item[{Returns}] \leavevmode
The figure.

\end{description}\end{quote}

\end{fulllineitems}

\index{getMetadata() (pyleoclim.SummaryPlots method)}

\begin{fulllineitems}
\phantomsection\label{SummaryPlots:pyleoclim.SummaryPlots.getMetadata}\pysiglinewithargsret{\bfcode{getMetadata}}{\emph{timeseries}}{}
Get the necessary metadata to be printed out automatically
\begin{quote}\begin{description}
\item[{Parameters}] \leavevmode
\textbf{\texttt{timeseries}} -- a specific timeseries object

\item[{Returns}] \leavevmode

archiveType

Authors (if more than 2, replace by et al.

PublicationYear

Publication DOI

Variable Name

Units

Climate Interpretation

Calibration Equation

Calibration References

Calibration Notes


\item[{Return type}] \leavevmode
A dictionary containing the following metadata

\end{description}\end{quote}

\end{fulllineitems}


\end{fulllineitems}



\chapter{Manipulating LiPD files}
\label{LIPDutils::doc}\label{LIPDutils:manipulating-lipd-files}
The following methods allows to manipulate LiPD files and objects.


\section{Creating directories and saving figures}
\label{LIPDutils:creating-directories-and-saving-figures}\index{createDir() (in module pyleoclim)}

\begin{fulllineitems}
\phantomsection\label{LIPDutils:pyleoclim.createDir}\pysiglinewithargsret{\code{pyleoclim.}\bfcode{createDir}}{\emph{path}, \emph{foldername}}{}
Create a new folder in a working directory

Create a new folder in a working directory to save outputs from Pyleoclim.
\begin{quote}\begin{description}
\item[{Parameters}] \leavevmode\begin{itemize}
\item {} 
\textbf{\texttt{path}} (\emph{\texttt{str}}) -- the path to the new folder.

\item {} 
\textbf{\texttt{foldername}} (\emph{\texttt{str}}) -- the name of the folder to be created

\end{itemize}

\item[{Returns}] \leavevmode
newdir - the full path to the new directory

\end{description}\end{quote}

\end{fulllineitems}

\index{saveFigure() (in module pyleoclim)}

\begin{fulllineitems}
\phantomsection\label{LIPDutils:pyleoclim.saveFigure}\pysiglinewithargsret{\code{pyleoclim.}\bfcode{saveFigure}}{\emph{name}, \emph{format='eps'}, \emph{dir='`}}{}
Save a figure

Save the figure in the directory. If not given, creates a folder in the
lipd.path directory.
\begin{quote}\begin{description}
\item[{Parameters}] \leavevmode\begin{itemize}
\item {} 
\textbf{\texttt{name}} (\emph{\texttt{str}}) -- name of the file

\item {} 
\textbf{\texttt{format}} (\emph{\texttt{str}}) -- One of the file extensions supported by the active
backend. Default is ``eps''. Most backend support png, pdf, ps, eps,
and svg.

\item {} 
\textbf{\texttt{dir}} (\emph{\texttt{str}}) -- the name of the folder in the LiPD working directory.
If not provided, creates a default folder called `figures'.

\end{itemize}

\end{description}\end{quote}

\end{fulllineitems}



\section{Manipulating LiPD files}
\label{LIPDutils:id1}\index{enumerateLipds() (in module pyleoclim)}

\begin{fulllineitems}
\phantomsection\label{LIPDutils:pyleoclim.enumerateLipds}\pysiglinewithargsret{\code{pyleoclim.}\bfcode{enumerateLipds}}{}{}
Enumerate the LiPD files loaded in the workspace

\end{fulllineitems}

\index{promptForLipd() (in module pyleoclim)}

\begin{fulllineitems}
\phantomsection\label{LIPDutils:pyleoclim.promptForLipd}\pysiglinewithargsret{\code{pyleoclim.}\bfcode{promptForLipd}}{}{}
Prompt for a LiPD file

Ask the user to select a LiPD file from a list
Use this function in conjunction with enumerateLipds()
\begin{quote}\begin{description}
\item[{Returns}] \leavevmode
The index of the LiPD file

\end{description}\end{quote}

\end{fulllineitems}



\section{Manipulating Variables in a LiPD file}
\label{LIPDutils:manipulating-variables-in-a-lipd-file}\index{promptForVariable() (in module pyleoclim)}

\begin{fulllineitems}
\phantomsection\label{LIPDutils:pyleoclim.promptForVariable}\pysiglinewithargsret{\code{pyleoclim.}\bfcode{promptForVariable}}{}{}
Prompt for a specific variable

Ask the user to select the variable they are interested in.
Use this function in conjunction with readHeaders() or getTSO()
\begin{quote}\begin{description}
\item[{Returns}] \leavevmode
The index of the variable

\end{description}\end{quote}

\end{fulllineitems}

\index{valuesLoc() (in module pyleoclim)}

\begin{fulllineitems}
\phantomsection\label{LIPDutils:pyleoclim.valuesLoc}\pysiglinewithargsret{\code{pyleoclim.}\bfcode{valuesLoc}}{\emph{dataframe}, \emph{missing\_value='NaN'}, \emph{var\_idx=1}}{}
Remove missing values flag

Look for the indexes where there are no missing values for the variable
\begin{quote}\begin{description}
\item[{Parameters}] \leavevmode\begin{itemize}
\item {} 
\textbf{\texttt{dataframe}} -- a Pandas Dataframe

\item {} 
\textbf{\texttt{missing\_value}} (\emph{\texttt{str or float}}) -- how are the missing value represented.
Default is NaN

\item {} 
\textbf{\texttt{var\_idx}} (\emph{\texttt{int}}) -- the column number in which to look for the missing
values (default is the second column)

\end{itemize}

\item[{Returns}] \leavevmode
val\_idx - the indices of the lines in the dataframe containing the actual values

\end{description}\end{quote}

\end{fulllineitems}

\index{xAxisTs() (in module pyleoclim)}

\begin{fulllineitems}
\phantomsection\label{LIPDutils:pyleoclim.xAxisTs}\pysiglinewithargsret{\code{pyleoclim.}\bfcode{xAxisTs}}{\emph{timeseries}}{}
Prompt the user to choose a x-axis representation for the timeseries.
\begin{quote}\begin{description}
\item[{Parameters}] \leavevmode
\textbf{\texttt{timeseries}} -- a timeseries object

\item[{Returns}] \leavevmode

x\_axis - the values for the x-axis representation,

label - returns either ``age'', ``year'', or ``depth''


\end{description}\end{quote}

\end{fulllineitems}



\section{Manipulating timeseries objects}
\label{LIPDutils:manipulating-timeseries-objects}\index{enumerateTs() (in module pyleoclim)}

\begin{fulllineitems}
\phantomsection\label{LIPDutils:pyleoclim.enumerateTs}\pysiglinewithargsret{\code{pyleoclim.}\bfcode{enumerateTs}}{\emph{timeseries\_list}}{}
Enumerate the available time series objects
\begin{quote}\begin{description}
\item[{Parameters}] \leavevmode
\textbf{\texttt{timeseries\_list}} -- a  list of available timeseries objects.
To use the timeseries loaded upon initiation of the
pyleoclim package, use pyleo.time\_series.

\end{description}\end{quote}

\end{fulllineitems}

\index{getTs() (in module pyleoclim)}

\begin{fulllineitems}
\phantomsection\label{LIPDutils:pyleoclim.getTs}\pysiglinewithargsret{\code{pyleoclim.}\bfcode{getTs}}{\emph{timeseries\_list}}{}
Get a specific timeseries object from a dictionary of timeseries
\begin{quote}\begin{description}
\item[{Parameters}] \leavevmode
\textbf{\texttt{timeseries\_list}} -- a  list of available timeseries objects.
To use the timeseries loaded upon initiation of the
pyleoclim package, use pyleo.time\_series.

\item[{Returns}] \leavevmode
A single timeseries object

\end{description}\end{quote}

\end{fulllineitems}

\index{TsToDf() (in module pyleoclim)}

\begin{fulllineitems}
\phantomsection\label{LIPDutils:pyleoclim.TsToDf}\pysiglinewithargsret{\code{pyleoclim.}\bfcode{TsToDf}}{\emph{timeseries}, \emph{x\_axis='`}}{}
Timeseries to Dataframe

Create a dataframe from a timeseries object with two colums:
depth/age representation and the paleoData values
\begin{quote}\begin{description}
\item[{Parameters}] \leavevmode\begin{itemize}
\item {} 
\textbf{\texttt{timeseries}} -- A timeseries object

\item {} 
\textbf{\texttt{x-axis}} (\emph{\texttt{str}}) -- The representation against which to plot the paleo-data.
Options are ``age'', ``year'', and ``depth''. Default is to let the
system choose if only one available or prompt the user.

\end{itemize}

\item[{Returns}] \leavevmode
A Pandas Dataframe with two columns - the x-axis representation
(``year'', ``age'', or ``depth'') and the PaleoDataValues

\end{description}\end{quote}

\end{fulllineitems}



\section{Handling mapping to LinkedEarth Ontology}
\label{LIPDutils:handling-mapping-to-linkedearth-ontology}\index{LipdToOntology() (in module pyleoclim)}

\begin{fulllineitems}
\phantomsection\label{LIPDutils:pyleoclim.LipdToOntology}\pysiglinewithargsret{\code{pyleoclim.}\bfcode{LipdToOntology}}{\emph{archiveType}}{}
standardize archiveType

Transform the archiveType from their LiPD name to their ontology counterpart
\begin{quote}\begin{description}
\item[{Parameters}] \leavevmode
\textbf{\texttt{archiveType}} (\emph{\texttt{STR}}) -- name of the archiveType from the LiPD file

\item[{Returns}] \leavevmode
archiveType according to the ontology

\end{description}\end{quote}

\end{fulllineitems}



\chapter{Indices and tables}
\label{index:indices-and-tables}\begin{itemize}
\item {} 
\DUspan{xref,std,std-ref}{genindex}

\item {} 
\DUspan{xref,std,std-ref}{modindex}

\item {} 
\DUspan{xref,std,std-ref}{search}

\end{itemize}



\renewcommand{\indexname}{Index}
\printindex
\end{document}
